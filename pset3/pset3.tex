% Options for packages loaded elsewhere
\PassOptionsToPackage{unicode}{hyperref}
\PassOptionsToPackage{hyphens}{url}
\PassOptionsToPackage{dvipsnames,svgnames,x11names}{xcolor}
%
\documentclass[
]{article}
\usepackage{amsmath,amssymb}
\usepackage{iftex}
\ifPDFTeX
  \usepackage[T1]{fontenc}
  \usepackage[utf8]{inputenc}
  \usepackage{textcomp} % provide euro and other symbols
\else % if luatex or xetex
  \usepackage{unicode-math} % this also loads fontspec
  \defaultfontfeatures{Scale=MatchLowercase}
  \defaultfontfeatures[\rmfamily]{Ligatures=TeX,Scale=1}
\fi
\usepackage{lmodern}
\ifPDFTeX\else
  % xetex/luatex font selection
\fi
% Use upquote if available, for straight quotes in verbatim environments
\IfFileExists{upquote.sty}{\usepackage{upquote}}{}
\IfFileExists{microtype.sty}{% use microtype if available
  \usepackage[]{microtype}
  \UseMicrotypeSet[protrusion]{basicmath} % disable protrusion for tt fonts
}{}
\makeatletter
\@ifundefined{KOMAClassName}{% if non-KOMA class
  \IfFileExists{parskip.sty}{%
    \usepackage{parskip}
  }{% else
    \setlength{\parindent}{0pt}
    \setlength{\parskip}{6pt plus 2pt minus 1pt}}
}{% if KOMA class
  \KOMAoptions{parskip=half}}
\makeatother
\usepackage{xcolor}
\usepackage[margin=1in]{geometry}
\usepackage{color}
\usepackage{fancyvrb}
\newcommand{\VerbBar}{|}
\newcommand{\VERB}{\Verb[commandchars=\\\{\}]}
\DefineVerbatimEnvironment{Highlighting}{Verbatim}{commandchars=\\\{\}}
% Add ',fontsize=\small' for more characters per line
\usepackage{framed}
\definecolor{shadecolor}{RGB}{248,248,248}
\newenvironment{Shaded}{\begin{snugshade}}{\end{snugshade}}
\newcommand{\AlertTok}[1]{\textcolor[rgb]{0.94,0.16,0.16}{#1}}
\newcommand{\AnnotationTok}[1]{\textcolor[rgb]{0.56,0.35,0.01}{\textbf{\textit{#1}}}}
\newcommand{\AttributeTok}[1]{\textcolor[rgb]{0.13,0.29,0.53}{#1}}
\newcommand{\BaseNTok}[1]{\textcolor[rgb]{0.00,0.00,0.81}{#1}}
\newcommand{\BuiltInTok}[1]{#1}
\newcommand{\CharTok}[1]{\textcolor[rgb]{0.31,0.60,0.02}{#1}}
\newcommand{\CommentTok}[1]{\textcolor[rgb]{0.56,0.35,0.01}{\textit{#1}}}
\newcommand{\CommentVarTok}[1]{\textcolor[rgb]{0.56,0.35,0.01}{\textbf{\textit{#1}}}}
\newcommand{\ConstantTok}[1]{\textcolor[rgb]{0.56,0.35,0.01}{#1}}
\newcommand{\ControlFlowTok}[1]{\textcolor[rgb]{0.13,0.29,0.53}{\textbf{#1}}}
\newcommand{\DataTypeTok}[1]{\textcolor[rgb]{0.13,0.29,0.53}{#1}}
\newcommand{\DecValTok}[1]{\textcolor[rgb]{0.00,0.00,0.81}{#1}}
\newcommand{\DocumentationTok}[1]{\textcolor[rgb]{0.56,0.35,0.01}{\textbf{\textit{#1}}}}
\newcommand{\ErrorTok}[1]{\textcolor[rgb]{0.64,0.00,0.00}{\textbf{#1}}}
\newcommand{\ExtensionTok}[1]{#1}
\newcommand{\FloatTok}[1]{\textcolor[rgb]{0.00,0.00,0.81}{#1}}
\newcommand{\FunctionTok}[1]{\textcolor[rgb]{0.13,0.29,0.53}{\textbf{#1}}}
\newcommand{\ImportTok}[1]{#1}
\newcommand{\InformationTok}[1]{\textcolor[rgb]{0.56,0.35,0.01}{\textbf{\textit{#1}}}}
\newcommand{\KeywordTok}[1]{\textcolor[rgb]{0.13,0.29,0.53}{\textbf{#1}}}
\newcommand{\NormalTok}[1]{#1}
\newcommand{\OperatorTok}[1]{\textcolor[rgb]{0.81,0.36,0.00}{\textbf{#1}}}
\newcommand{\OtherTok}[1]{\textcolor[rgb]{0.56,0.35,0.01}{#1}}
\newcommand{\PreprocessorTok}[1]{\textcolor[rgb]{0.56,0.35,0.01}{\textit{#1}}}
\newcommand{\RegionMarkerTok}[1]{#1}
\newcommand{\SpecialCharTok}[1]{\textcolor[rgb]{0.81,0.36,0.00}{\textbf{#1}}}
\newcommand{\SpecialStringTok}[1]{\textcolor[rgb]{0.31,0.60,0.02}{#1}}
\newcommand{\StringTok}[1]{\textcolor[rgb]{0.31,0.60,0.02}{#1}}
\newcommand{\VariableTok}[1]{\textcolor[rgb]{0.00,0.00,0.00}{#1}}
\newcommand{\VerbatimStringTok}[1]{\textcolor[rgb]{0.31,0.60,0.02}{#1}}
\newcommand{\WarningTok}[1]{\textcolor[rgb]{0.56,0.35,0.01}{\textbf{\textit{#1}}}}
\usepackage{graphicx}
\makeatletter
\def\maxwidth{\ifdim\Gin@nat@width>\linewidth\linewidth\else\Gin@nat@width\fi}
\def\maxheight{\ifdim\Gin@nat@height>\textheight\textheight\else\Gin@nat@height\fi}
\makeatother
% Scale images if necessary, so that they will not overflow the page
% margins by default, and it is still possible to overwrite the defaults
% using explicit options in \includegraphics[width, height, ...]{}
\setkeys{Gin}{width=\maxwidth,height=\maxheight,keepaspectratio}
% Set default figure placement to htbp
\makeatletter
\def\fps@figure{htbp}
\makeatother
\setlength{\emergencystretch}{3em} % prevent overfull lines
\providecommand{\tightlist}{%
  \setlength{\itemsep}{0pt}\setlength{\parskip}{0pt}}
\setcounter{secnumdepth}{-\maxdimen} % remove section numbering
\usepackage{amsmath}
\ifLuaTeX
  \usepackage{selnolig}  % disable illegal ligatures
\fi
\usepackage{bookmark}
\IfFileExists{xurl.sty}{\usepackage{xurl}}{} % add URL line breaks if available
\urlstyle{same}
\hypersetup{
  pdftitle={Problem Set 3},
  pdfauthor={Tessie Dong, Derek Li, Andi Liu},
  colorlinks=true,
  linkcolor={Maroon},
  filecolor={Maroon},
  citecolor={Blue},
  urlcolor={Blue},
  pdfcreator={LaTeX via pandoc}}

\title{Problem Set 3}
\author{Tessie Dong, Derek Li, Andi Liu}
\date{Due Jan 18th, 2024}

\begin{document}
\maketitle

\textbf{Part 1: Describe the Data} \newline Combine the NSW data in
\texttt{nswre74\_control.csv} and \texttt{nswre74\_treated.csv} and
complete Table \ref{tab:Tab_NSW_1}. Note that variables \texttt{1-10}
are \(\color{ForestGreen}{predetermined}\), i.e., capture
characteristics determined at or before treatment assignment; some of
these variables are background characteristics (e.g., \texttt{edu}),
others capture a subject's pre-RCT labor market experience (e.g.,
\texttt{u75}). \texttt{re78} is the observed outcome variable.
\texttt{treat} is the indicator of treatment status.

\begin{Shaded}
\begin{Highlighting}[]
\CommentTok{\# load and combine the data sets into one dataframe}
\NormalTok{df1 }\OtherTok{\textless{}{-}}\NormalTok{ utils}\SpecialCharTok{::}\FunctionTok{read.csv}\NormalTok{(}\AttributeTok{file =} \StringTok{"nswre74\_control.csv"}\NormalTok{)}
\NormalTok{df2 }\OtherTok{\textless{}{-}}\NormalTok{ utils}\SpecialCharTok{::}\FunctionTok{read.csv}\NormalTok{(}\AttributeTok{file =} \StringTok{"nswre74\_treated.csv"}\NormalTok{)}
\NormalTok{df }\OtherTok{\textless{}{-}} \FunctionTok{rbind}\NormalTok{(df1, df2)}
\end{Highlighting}
\end{Shaded}

\begin{Shaded}
\begin{Highlighting}[]
\CommentTok{\# count units in each sample}
\NormalTok{dplyr}\SpecialCharTok{::}\FunctionTok{tally}\NormalTok{(dplyr}\SpecialCharTok{::}\FunctionTok{group\_by}\NormalTok{(df, treat))}
\end{Highlighting}
\end{Shaded}

\begin{verbatim}
## # A tibble: 2 x 2
##   treat     n
##   <int> <int>
## 1     0   260
## 2     1   185
\end{verbatim}

\begin{Shaded}
\begin{Highlighting}[]
\CommentTok{\# generate mean summary statistics for each variable and treatment}
\NormalTok{dplyr}\SpecialCharTok{::}\FunctionTok{summarise\_all}\NormalTok{((dplyr}\SpecialCharTok{::}\FunctionTok{group\_by}\NormalTok{(df, treat)), }\FunctionTok{list}\NormalTok{(mean))}
\end{Highlighting}
\end{Shaded}

\begin{verbatim}
## # A tibble: 2 x 12
##   treat   age   edu black   hisp married nodegree  re74  re75  re78   u74   u75
##   <int> <dbl> <dbl> <dbl>  <dbl>   <dbl>    <dbl> <dbl> <dbl> <dbl> <dbl> <dbl>
## 1     0  25.1  10.1 0.827 0.108    0.154    0.835 2107. 1267. 4555. 0.75  0.685
## 2     1  25.8  10.3 0.843 0.0595   0.189    0.708 2096. 1532. 6349. 0.708 0.6
\end{verbatim}

A completed version of Table 1 is provided on the following page.
\newpage

\begin{table}[!ht!]
\centering
\begin{tabular}{ccccc}
\hline
\textbf{Variable} & \textbf{Variable}    & \textbf{Variable}    & \multicolumn{2}{c}{\textbf{Sample Average}}    \\ \cline{4-5} 
\textbf{Counter} & \textbf{Name}       & \textbf{Definition}                        & \textbf{Treated} & \textbf{Control} \\ \hline
1 & \texttt{age}      & Age in years                      & 25.8        & 25.1       \\
2 & \texttt{edu}      & Education in years                & 10.3        & 10.1       \\
3 & \texttt{nodegree} & 1 if education $<12$              & 0.708       & 0.835      \\
4 & \texttt{black}    & 1 if Black                        & 0.843       & 0.827      \\
5 & \texttt{hisp}     & 1 if Hispanic                     & 0.0595      & 0.108      \\
6 & \texttt{married}  & 1 if married                      & 0.189       & 0.154      \\
7 & \texttt{u74}      & 1 if unemployed in '74            & 0.708       & 0.75       \\
8 & \texttt{u75}      & 1 if unemployed in '75            & 0.6         & 0.685      \\
9 & \texttt{re74}     & Real earnings in '74 (in '82 \$)  & 2096        & 2107       \\
10 & \texttt{re75}    & Real earnings in '75 (in '82 \$)  & 1532        & 1267       \\
\hline 
11 & \texttt{re78}    & Real earnings in '78 (in '82 \$)  & 6349        & 4555       \\
12 & \texttt{treat}   & 1 if received offer of training   & 1           & 0          \\
\hline
Sample Size                                             &&& 185         & 260              \\ \hline
\end{tabular}
\caption{Descriptive statistics for the NSW data by group.}
\label{tab:Tab_NSW_1}
\end{table}

\textbf{Part 2: Test Balance}

\begin{enumerate}
\item (5 p) Test \textcolor{ForestGreen}{balance} for each of the 10 OPVs in Table (\ref{tab:Tab_NSW_1}), i.e.,  test that each variable's mean is the same in the control and treated groups. Do so by running 10 simple linear regressions (SLR) specifications. Use a 5\% significance level and look at the relevant 10 t-tests, comment on your findings. \textcolor{gray}{\textbf{Hint}: Each OPV is the dependent variable in its regression equation. All 10 SLR models have the same covariates.}

\item (40 p) Testing balance as done in Q1 suffers from the so called \href{https://en.wikipedia.org/wiki/Multiple_comparisons_problem}{\textcolor{ForestGreen}{``multiple comparisons''}} or \textcolor{ForestGreen}{``multiple testing'' problem} which occurs when one considers a set of statistical inferences simultaneously. The \href{https://explainxkcd.com/wiki/index.php/882:_Significant}{problem} emerges because as more variables are compared, it becomes more likely that the treatment and control groups appear to differ on at least one attribute \textit{by random chance alone}. To deal
with this problem we use an estimation methodology called \textcolor{ForestGreen}{SUR estimation} and then test just one hypothesis, the \textit{joint} hypothesis that all OPVs are balanced, i.e., their means are the same in the two groups. SUR stands for \textcolor{ForestGreen}{seemingly unrelated regression} and it is a special case of \textcolor{ForestGreen}{feasible Generalized Least Squares (GLS) estimation}. Instead of estimating the coefficients \textit{equation-by-equation} by OLS (and done in Q1 you combine the 10 equations in a system of equations and estimate the coefficients present in all the equations jointly, accounting for the fact that the unobservables may be correlated across equations within an individual (we continue to assume that they are uncorrelated across units). After estimation, you use standard testing procedures to test the \textit{joint} hypothesis that the slope coefficients in all the equations of the SUR system are zero. This joint test is a test of covariate balance that does not suffer from the ``multiple testing'' problem.

\begin{enumerate}
\item (25 p) Estimate the SUR system. Are the estimated coefficients and their SEs different from those obtained in Q\textbf{\ref{item:balance_test:single_tests}}? Comment. \textcolor{gray}{\textbf{Hint}: In 2 situations there is no efficiency payoff to GLS versus OLS: 1) the unobservables are uncorrelated across equations within an individual; and 2) the equations have identical covariates.}

\item (15 p) Test \textit{joint} balance by testing the \textcolor{ForestGreen}{joint hypothesis} that the coefficients of \texttt{treat} are zero in all the equations of the system. Spell out null and alternative hypotheses, comment on findings, and verify the test's value and p-value ``manually.'' \textcolor{gray}{\textbf{Hint}: Use the \href{https://en.wikipedia.org/wiki/Likelihood-ratio_test}{\textcolor{ForestGreen}{Likelihood Ratio (LR) test}} where the unrestricted model is the system of equations with \texttt{treat} as covariate, and the restricted model is the system with no regression covariates, i.e., with only the constant term. To verify the value of the test use its algebraic expression. Recall that the test has a $\chi^{2}_{df}$ distribution with $df$ = number of restrictions.}
\end{enumerate}

\item (10 p) Test that the OPVs do not predict treatment assignment.  Spell out null and alternative hypotheses, comment on findings, and verify the test's value and p-value ``manually.'' Why would scientists carry out this test? \textcolor{gray}{\textbf{Hint:} Use a MLRM and the \textcolor{ForestGreen}{F-test for the overall significance of the regression}. \textbf{Programming guidance}: Use \texttt{summary()} after estimation, e.g., \texttt{summary(lm\_fit)\$fstatistic} returns the test's value and degrees of freedom. Use \href{https://stat.ethz.ch/R-manual/R-devel/library/stats/html/Fdist.html}{\texttt{stats::pf()}} to verify the test's p-value.}\label{item:balance_test:predict_treat} 
\end{enumerate}

\end{document}
