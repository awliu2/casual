% Options for packages loaded elsewhere
\PassOptionsToPackage{unicode}{hyperref}
\PassOptionsToPackage{hyphens}{url}
\PassOptionsToPackage{dvipsnames,svgnames,x11names}{xcolor}
%
\documentclass[
]{article}
\usepackage{amsmath,amssymb}
\usepackage{iftex}
\ifPDFTeX
  \usepackage[T1]{fontenc}
  \usepackage[utf8]{inputenc}
  \usepackage{textcomp} % provide euro and other symbols
\else % if luatex or xetex
  \usepackage{unicode-math} % this also loads fontspec
  \defaultfontfeatures{Scale=MatchLowercase}
  \defaultfontfeatures[\rmfamily]{Ligatures=TeX,Scale=1}
\fi
\usepackage{lmodern}
\ifPDFTeX\else
  % xetex/luatex font selection
\fi
% Use upquote if available, for straight quotes in verbatim environments
\IfFileExists{upquote.sty}{\usepackage{upquote}}{}
\IfFileExists{microtype.sty}{% use microtype if available
  \usepackage[]{microtype}
  \UseMicrotypeSet[protrusion]{basicmath} % disable protrusion for tt fonts
}{}
\makeatletter
\@ifundefined{KOMAClassName}{% if non-KOMA class
  \IfFileExists{parskip.sty}{%
    \usepackage{parskip}
  }{% else
    \setlength{\parindent}{0pt}
    \setlength{\parskip}{6pt plus 2pt minus 1pt}}
}{% if KOMA class
  \KOMAoptions{parskip=half}}
\makeatother
\usepackage{xcolor}
\usepackage[margin=1cm]{geometry}
\usepackage{graphicx}
\makeatletter
\def\maxwidth{\ifdim\Gin@nat@width>\linewidth\linewidth\else\Gin@nat@width\fi}
\def\maxheight{\ifdim\Gin@nat@height>\textheight\textheight\else\Gin@nat@height\fi}
\makeatother
% Scale images if necessary, so that they will not overflow the page
% margins by default, and it is still possible to overwrite the defaults
% using explicit options in \includegraphics[width, height, ...]{}
\setkeys{Gin}{width=\maxwidth,height=\maxheight,keepaspectratio}
% Set default figure placement to htbp
\makeatletter
\def\fps@figure{htbp}
\makeatother
\setlength{\emergencystretch}{3em} % prevent overfull lines
\providecommand{\tightlist}{%
  \setlength{\itemsep}{0pt}\setlength{\parskip}{0pt}}
\setcounter{secnumdepth}{-\maxdimen} % remove section numbering
\usepackage{amsmath}
\usepackage{optidef}
\usepackage{accents}
\usepackage{caption}
\ifLuaTeX
  \usepackage{selnolig}  % disable illegal ligatures
\fi
\IfFileExists{bookmark.sty}{\usepackage{bookmark}}{\usepackage{hyperref}}
\IfFileExists{xurl.sty}{\usepackage{xurl}}{} % add URL line breaks if available
\urlstyle{same}
\hypersetup{
  pdftitle={Problem Set 8: Estimation of TEs with Instrumental Variables},
  pdfauthor={Tessie Dong, Derek Li, Andi Liu},
  colorlinks=true,
  linkcolor={Maroon},
  filecolor={Maroon},
  citecolor={Blue},
  urlcolor={Blue},
  pdfcreator={LaTeX via pandoc}}

\title{Problem Set 8: Estimation of TEs with Instrumental Variables}
\author{Tessie Dong, Derek Li, Andi Liu}
\date{Due Feb 29th, 2024}

\begin{document}
\maketitle

\hypertarget{part-1-the-wald-estimator}{%
\subsection{Part 1: The Wald
Estimator}\label{part-1-the-wald-estimator}}

\hypertarget{part-2-guido-imbens-2021-economics-nobel-prize-lecture}{%
\subsection{Part 2: Guido Imbens' 2021 Economics Nobel Prize
Lecture}\label{part-2-guido-imbens-2021-economics-nobel-prize-lecture}}

\begin{enumerate}
\def\labelenumi{\arabic{enumi}.}
\setcounter{enumi}{7}
\tightlist
\item
  (0.5 p) The fundamental problem of causal inference.
\end{enumerate}

3:10

\begin{enumerate}
\def\labelenumi{\arabic{enumi}.}
\setcounter{enumi}{8}
\tightlist
\item
  (0.5 p) That random control trials (RCTs) ensure balance.
\end{enumerate}

4:19

\begin{enumerate}
\def\labelenumi{\arabic{enumi}.}
\setcounter{enumi}{9}
\tightlist
\item
  (0.5 p) Observational studies and the problem of confounders.
\end{enumerate}

4:37

\begin{enumerate}
\def\labelenumi{\arabic{enumi}.}
\setcounter{enumi}{10}
\tightlist
\item
  (0.5 p) Rubin's Potential Outcome Model also called Rubin Causal Model
  (RCM).
\end{enumerate}

8:02

\begin{enumerate}
\def\labelenumi{\arabic{enumi}.}
\setcounter{enumi}{11}
\tightlist
\item
  (0.5 p) The NSW experiment. Note: you used this data in several PSets.
\end{enumerate}

10:17

\begin{enumerate}
\def\labelenumi{\arabic{enumi}.}
\setcounter{enumi}{12}
\tightlist
\item
  (0.5 p) The ``observational'' data constructed by combining the NSW
  experimental sample with a control sample taken from survey data such
  as PSID. Note: You used this data in several PSets.
\end{enumerate}

10:38

\begin{enumerate}
\def\labelenumi{\arabic{enumi}.}
\setcounter{enumi}{13}
\tightlist
\item
  (0.5 p) Matching and propensity score method. Note: You applied these
  methods in several PSets.
\end{enumerate}

11:50

\begin{enumerate}
\def\labelenumi{\arabic{enumi}.}
\setcounter{enumi}{14}
\tightlist
\item
  (0.5 p) The 2001 Imbens, Rubin, Sacerdote's Lottery paper. Note: We
  discuss this paper in CAUS 3.pdf.
\end{enumerate}

12:42

\begin{enumerate}
\def\labelenumi{\arabic{enumi}.}
\setcounter{enumi}{15}
\tightlist
\item
  (0.5 p) The difference in differences (DD) estimation method.
\end{enumerate}

26:09

\begin{enumerate}
\def\labelenumi{\arabic{enumi}.}
\setcounter{enumi}{16}
\tightlist
\item
  (0.5 p) Void. Note: This half point is assigned to everybody by
  default.
\end{enumerate}

\hypertarget{part-3-part-3-joshua-angrist-2021-economics-nobel-prize-lecture}{%
\subsection{Part 3: Part 3: Joshua Angrist' 2021 Economics Nobel Prize
Lecture}\label{part-3-part-3-joshua-angrist-2021-economics-nobel-prize-lecture}}

\begin{enumerate}
\def\labelenumi{\arabic{enumi}.}
\setcounter{enumi}{17}
\tightlist
\item
  (1 p) What is Angrist's definition of ``empirical strategy?''
\end{enumerate}

He defined empirical strategy to be a research plan that emcompasses
data collection, identification, and econometric estimation.

\begin{enumerate}
\def\labelenumi{\arabic{enumi}.}
\setcounter{enumi}{18}
\tightlist
\item
  (5 p) Angrist's describes RD via multiple applications. Understand the
  approach and summarize it in 1 paragraph written to explain the
  approach to your classmates. Make sure to briefly explain why Angrist
  says that ``Fuzzy RD is IV.''
\end{enumerate}

RD is an approach that exploits the jumps in human affairs induced by
rules and the need to classify people for various assignment purposes.
An application that he mentions is the Nobel prize nomination which
involves a tiebreaking variable crossing or failing to cross a threshold
(becoming a Nobel prize winner or just missing it - ``near Nobel''). In
these situations, those just below the threshold become a natural
control group for those who clear it. To further explain Angrist's claim
that ``Fuzzy ID is IV'', Fuzzy RD leverages discontinuities in the
probability or expected value of treatment based on a covariate. This
creates a study framework where the discontinuity serves as an
instrumental variable for treatment status, rather than simply toggling
treatment on or off deterministically.

\begin{enumerate}
\def\labelenumi{\arabic{enumi}.}
\setcounter{enumi}{19}
\tightlist
\item
  (2 p) Compare and contrast the empirical strategy of Alan Krueger's
  1999 QJE article that estimate the impact of class size on test scores
  in Tennessee's elementary schools using data from the STAR RCT to the
  empirical strategy of Angrist and Lavy's 1999 QJE article ``Using
  Maimonides' rule to estimate the effect of class size on scholastic
  achievement'' that estimates the impact of class size in Israel's
  elementary schools
\end{enumerate}

Krueger's analysis draws upon data from the STAR project, a large-scale
RCT. Within this trial, students were randomly assigned to small classes
consisting of 13-17 students, regular classes with 22-25 students, or
regular classes supplemented with a teacher's aide. This random
assignment serves to mitigate selection bias and establishes a clear
causal relationship between class size and academic outcomes. On the
other hand, Angrist and Lavy employ a natural experiment rooted in
Maimonides' Rule, a historical guideline governing Israeli schools that
sets a maximum class size of 40 students. When student enrollment
surpasses this threshold, an additional class is formed. These imposed
thresholds create discontinuities in class size, which the authors
utilize as an instrumental variable to gauge the impact of class size on
academic achievement.

\begin{enumerate}
\def\labelenumi{\arabic{enumi}.}
\setcounter{enumi}{20}
\tightlist
\item
  (2 p) Angrist asks ``What is the causal effect of charter school
  attendance on learning?'' Explain the implications of the ``LATE
  theorem'' for answering this question.
\end{enumerate}

Addressing this inquiry involves dealing with issues of selection bias
and endogeneity. Students who enroll in charter schools may exhibit
systematic differences from those who do not, not only in observable
traits but also in unobservable factors like motivation or parental
involvement. The LATE theorem presents a robust framework for tackling
this issue, especially in scenarios where the treatment, such as
attending a charter school, is not randomly assigned. In attempting to
answer this question, the LATE theorem provides a rigorous strategy to
estimate the causal effect of charter school attendance on academic
performance for a specific subgroup of students, thereby confronting the
challenges posed by selection bias and endogeneity.

The LATE theorem offers a means to estimate the causal impact of a
treatment on a specific subset of the population, known as
``compliers,'' even in the presence of endogenous selection. It pertains
to individuals who adhere to the treatment assignment within a context
featuring an IV. In the case of charter schools, the IV was winning the
charter school lottery. Compliers, in this context, are students who opt
for charter schools because of lottery success but would have chosen
traditional public schools otherwise.The effectiveness of this approach
hinges on the validity of the IV --- it must correlate with the
likelihood of attending a charter school (relevance) and should not
directly influence the outcome variable except through its impact on the
treatment (exogeneity).

\hypertarget{part-4-compliers}{%
\subsection{Part 4: Compliers}\label{part-4-compliers}}

\end{document}
