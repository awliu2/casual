% Options for packages loaded elsewhere
\PassOptionsToPackage{unicode}{hyperref}
\PassOptionsToPackage{hyphens}{url}
\PassOptionsToPackage{dvipsnames,svgnames,x11names}{xcolor}
%
\documentclass[
]{article}
\usepackage{amsmath,amssymb}
\usepackage{iftex}
\ifPDFTeX
  \usepackage[T1]{fontenc}
  \usepackage[utf8]{inputenc}
  \usepackage{textcomp} % provide euro and other symbols
\else % if luatex or xetex
  \usepackage{unicode-math} % this also loads fontspec
  \defaultfontfeatures{Scale=MatchLowercase}
  \defaultfontfeatures[\rmfamily]{Ligatures=TeX,Scale=1}
\fi
\usepackage{lmodern}
\ifPDFTeX\else
  % xetex/luatex font selection
\fi
% Use upquote if available, for straight quotes in verbatim environments
\IfFileExists{upquote.sty}{\usepackage{upquote}}{}
\IfFileExists{microtype.sty}{% use microtype if available
  \usepackage[]{microtype}
  \UseMicrotypeSet[protrusion]{basicmath} % disable protrusion for tt fonts
}{}
\makeatletter
\@ifundefined{KOMAClassName}{% if non-KOMA class
  \IfFileExists{parskip.sty}{%
    \usepackage{parskip}
  }{% else
    \setlength{\parindent}{0pt}
    \setlength{\parskip}{6pt plus 2pt minus 1pt}}
}{% if KOMA class
  \KOMAoptions{parskip=half}}
\makeatother
\usepackage{xcolor}
\usepackage[margin=1in]{geometry}
\usepackage{color}
\usepackage{fancyvrb}
\newcommand{\VerbBar}{|}
\newcommand{\VERB}{\Verb[commandchars=\\\{\}]}
\DefineVerbatimEnvironment{Highlighting}{Verbatim}{commandchars=\\\{\}}
% Add ',fontsize=\small' for more characters per line
\usepackage{framed}
\definecolor{shadecolor}{RGB}{248,248,248}
\newenvironment{Shaded}{\begin{snugshade}}{\end{snugshade}}
\newcommand{\AlertTok}[1]{\textcolor[rgb]{0.94,0.16,0.16}{#1}}
\newcommand{\AnnotationTok}[1]{\textcolor[rgb]{0.56,0.35,0.01}{\textbf{\textit{#1}}}}
\newcommand{\AttributeTok}[1]{\textcolor[rgb]{0.13,0.29,0.53}{#1}}
\newcommand{\BaseNTok}[1]{\textcolor[rgb]{0.00,0.00,0.81}{#1}}
\newcommand{\BuiltInTok}[1]{#1}
\newcommand{\CharTok}[1]{\textcolor[rgb]{0.31,0.60,0.02}{#1}}
\newcommand{\CommentTok}[1]{\textcolor[rgb]{0.56,0.35,0.01}{\textit{#1}}}
\newcommand{\CommentVarTok}[1]{\textcolor[rgb]{0.56,0.35,0.01}{\textbf{\textit{#1}}}}
\newcommand{\ConstantTok}[1]{\textcolor[rgb]{0.56,0.35,0.01}{#1}}
\newcommand{\ControlFlowTok}[1]{\textcolor[rgb]{0.13,0.29,0.53}{\textbf{#1}}}
\newcommand{\DataTypeTok}[1]{\textcolor[rgb]{0.13,0.29,0.53}{#1}}
\newcommand{\DecValTok}[1]{\textcolor[rgb]{0.00,0.00,0.81}{#1}}
\newcommand{\DocumentationTok}[1]{\textcolor[rgb]{0.56,0.35,0.01}{\textbf{\textit{#1}}}}
\newcommand{\ErrorTok}[1]{\textcolor[rgb]{0.64,0.00,0.00}{\textbf{#1}}}
\newcommand{\ExtensionTok}[1]{#1}
\newcommand{\FloatTok}[1]{\textcolor[rgb]{0.00,0.00,0.81}{#1}}
\newcommand{\FunctionTok}[1]{\textcolor[rgb]{0.13,0.29,0.53}{\textbf{#1}}}
\newcommand{\ImportTok}[1]{#1}
\newcommand{\InformationTok}[1]{\textcolor[rgb]{0.56,0.35,0.01}{\textbf{\textit{#1}}}}
\newcommand{\KeywordTok}[1]{\textcolor[rgb]{0.13,0.29,0.53}{\textbf{#1}}}
\newcommand{\NormalTok}[1]{#1}
\newcommand{\OperatorTok}[1]{\textcolor[rgb]{0.81,0.36,0.00}{\textbf{#1}}}
\newcommand{\OtherTok}[1]{\textcolor[rgb]{0.56,0.35,0.01}{#1}}
\newcommand{\PreprocessorTok}[1]{\textcolor[rgb]{0.56,0.35,0.01}{\textit{#1}}}
\newcommand{\RegionMarkerTok}[1]{#1}
\newcommand{\SpecialCharTok}[1]{\textcolor[rgb]{0.81,0.36,0.00}{\textbf{#1}}}
\newcommand{\SpecialStringTok}[1]{\textcolor[rgb]{0.31,0.60,0.02}{#1}}
\newcommand{\StringTok}[1]{\textcolor[rgb]{0.31,0.60,0.02}{#1}}
\newcommand{\VariableTok}[1]{\textcolor[rgb]{0.00,0.00,0.00}{#1}}
\newcommand{\VerbatimStringTok}[1]{\textcolor[rgb]{0.31,0.60,0.02}{#1}}
\newcommand{\WarningTok}[1]{\textcolor[rgb]{0.56,0.35,0.01}{\textbf{\textit{#1}}}}
\usepackage{graphicx}
\makeatletter
\def\maxwidth{\ifdim\Gin@nat@width>\linewidth\linewidth\else\Gin@nat@width\fi}
\def\maxheight{\ifdim\Gin@nat@height>\textheight\textheight\else\Gin@nat@height\fi}
\makeatother
% Scale images if necessary, so that they will not overflow the page
% margins by default, and it is still possible to overwrite the defaults
% using explicit options in \includegraphics[width, height, ...]{}
\setkeys{Gin}{width=\maxwidth,height=\maxheight,keepaspectratio}
% Set default figure placement to htbp
\makeatletter
\def\fps@figure{htbp}
\makeatother
\setlength{\emergencystretch}{3em} % prevent overfull lines
\providecommand{\tightlist}{%
  \setlength{\itemsep}{0pt}\setlength{\parskip}{0pt}}
\setcounter{secnumdepth}{-\maxdimen} % remove section numbering
\usepackage{amsmath}
\ifLuaTeX
  \usepackage{selnolig}  % disable illegal ligatures
\fi
\IfFileExists{bookmark.sty}{\usepackage{bookmark}}{\usepackage{hyperref}}
\IfFileExists{xurl.sty}{\usepackage{xurl}}{} % add URL line breaks if available
\urlstyle{same}
\hypersetup{
  pdftitle={Problem Set 4},
  pdfauthor={Tessie Dong, Derek Li, Andi Liu},
  colorlinks=true,
  linkcolor={Maroon},
  filecolor={Maroon},
  citecolor={Blue},
  urlcolor={Blue},
  pdfcreator={LaTeX via pandoc}}

\title{Problem Set 4}
\author{Tessie Dong, Derek Li, Andi Liu}
\date{Due Jan 26th, 2024}

\begin{document}
\maketitle

\begin{center}
{\LARGE Part 1: Describe the Data (10 p)}
\end{center}

\begin{enumerate}
\def\labelenumi{\arabic{enumi}.}
\item
  Fill Table \ref{tab:descriptive-stats}'s columns 5 and 6 using,
  respectively, the data in \texttt{nswpsid.csv} and in
  \texttt{nswcps.csv}.
  \textcolor{Gray}{\textbf{Notes}: You want to limit attention to observations with \texttt{treat=0}. You filled columns 3 and 4 in PSet 3.}

\begin{Shaded}
\begin{Highlighting}[]
\CommentTok{\# Load data}
\NormalTok{nswpsid }\OtherTok{\textless{}{-}} \FunctionTok{read\_csv}\NormalTok{(}\StringTok{"starter{-}files/nswpsid.csv"}\NormalTok{)}
\NormalTok{nswcps }\OtherTok{\textless{}{-}} \FunctionTok{read\_csv}\NormalTok{(}\StringTok{"starter{-}files/nswcps.csv"}\NormalTok{)}
\NormalTok{nswpsid\_treat0 }\OtherTok{\textless{}{-}}\NormalTok{ nswpsid }\SpecialCharTok{\%\textgreater{}\%} \FunctionTok{filter}\NormalTok{(treat }\SpecialCharTok{==} \DecValTok{0}\NormalTok{)}
\NormalTok{nswcps\_treat0 }\OtherTok{\textless{}{-}}\NormalTok{ nswcps }\SpecialCharTok{\%\textgreater{}\%} \FunctionTok{filter}\NormalTok{(treat }\SpecialCharTok{==} \DecValTok{0}\NormalTok{)}
\end{Highlighting}
\end{Shaded}

\begin{Shaded}
\begin{Highlighting}[]
\NormalTok{summary\_cps }\OtherTok{\textless{}{-}} \FunctionTok{summarise\_all}\NormalTok{(nswcps\_treat0, }\FunctionTok{list}\NormalTok{(mean))}
\NormalTok{summary\_psid }\OtherTok{\textless{}{-}} \FunctionTok{summarise\_all}\NormalTok{(nswpsid\_treat0, }\FunctionTok{list}\NormalTok{(mean))}
\end{Highlighting}
\end{Shaded}

  \begin{table}[ht!]
    \label{tab:descriptive-stats}
    \centering
    \begin{tabular}{cccccc}
    \hline
    \textbf{Variable} & \textbf{Definition} & \multicolumn{2}{c}{\textbf{NSW}}    & \textbf{PSID-1}  & \textbf{CPS-1}   \\ \cline{3-6} 
    \textbf{}         & \textbf{}           & \textbf{Treated} & \textbf{Control} & \textbf{Control} & \textbf{Control} \\ 
    \hline
    [1] & [2] & [3] & [4] & [5] & [6] \\ \hline
    \texttt{age}      & Age in years                     & 25.82 & 25.05 & 34.85 & 33.22  \\
    \texttt{edu}      & Education in years               & 10.35 & 10.09 & 12.12 & 12.03  \\
    \texttt{nodegree} & 1 if education $<12$             & 0.71  & 0.83  & 0.31  & 0.30   \\
    \texttt{black}    & 1 if Black                       & 0.84  & 0.83  & 0.25  & 0.07   \\
    \texttt{hisp}     & 1 if Hispanic                    & 0.06  & 0.11  & 0.03  & 0.07   \\
    \texttt{married}  & 1 if married                     & 0.19  & 0.15  & 0.87  & 0.71   \\
    \texttt{u74}      & 1 if unemployed in '74           & 0.71  & 0.75  & 0.09  & 0.12   \\
    \texttt{u75}      & 1 if unemployed in '75           & 0.60  & 0.68  & 0.10  & 0.11   \\
    \texttt{re74}     & Real earnings in '74 (in '82 \$) & 2,096 & 2,107 & 19429 & 14017  \\
    \texttt{re75}     & Real earnings in '75 (in '82 \$) & 1,532 & 1,267 & 19063 & 13631  \\
    \hline
    \texttt{re78}     & Real earnings in '78 (in '82 \$) & 6,349 & 4,555 & 21554   & 14847       \\
    \texttt{treat}    & 1 if received offer of training  & 1     & 0     & 0      & 0       \\ \hline
    Sample Size                        &                                  & 185   & 260   & 2,490 & 15,992 \\ \hline
    \end{tabular}
    \caption{Sample averages for the NSW data (treated and control groups), PSID-1 data, and CPI-1 data.}
    \end{table}

\item
  Briefly comment on the completed Table \ref{tab:descriptive-stats}.
  \textcolor{gray}{\textbf{Hint}: Are the PSID-1 and CPS-1 samples "good" control groups?}

  I would argue that these samples are not the best control groups -
  this is mostly because many of the OPV covariates from the PSID and
  CPS exhibit large differences from the characteristics of the NSW
  sample. For example, the average age of the NSW sample is 25.82, while
  the average age of the PSID sample is 34.5, and there are large
  differences in income across the three samples. This suggests that the
  populations from which PSID and CPS were drawn are not very similar to
  the population of the NSW sample - making comparisons between treated
  individuals in the NSW sample and ``untreated'' individuals in the
  PSID and CPS samples less reliable, in our opinion.
\item
  Why do you think that Dehajia and Wahba constructed their
  ``observational datasets'' by pulling together the treated sample from
  NSW and a sample of individuals drawn from either the PSID or the CPS
  data?
  \textcolor{gray}{\textbf{Hint:} Both PSID and CPS include information on whether an individual enrolled in a training course during the previous 12 months. Thus, Dehajia and Wahba could have exploited exclusively observational variation in whether an individual enrolled in a training program. Why do you think that they chose not to follow this approach?}

  We believe that Dehajia and Wahba chose to pool the NSW and PSID/CPS
  datasets because they wanted to have a larger sample size to work
  with. This is because the NSW sample is relatively small, and the
  PSID/CPS samples are much larger. By pooling the NSW and PSID/CPS
  samples, Dehajia and Wahba are able to increase the sample size of
  their dataset. In addition, by analyzing samples drawn from different
  distributions (i.e.~PSID/CPS datasets), they could increase the
  generalizability of their results to the population.
\end{enumerate}

\end{document}
