% Options for packages loaded elsewhere
\PassOptionsToPackage{unicode}{hyperref}
\PassOptionsToPackage{hyphens}{url}
\PassOptionsToPackage{dvipsnames,svgnames,x11names}{xcolor}
%
\documentclass[
]{article}
\title{Problem Set 7: Estimation of TEs Using Matching Methods}
\author{Tessie Dong, Derek Li, Andi Liu}
\date{Due Feb 22nd, 2024}

\usepackage{amsmath,amssymb}
\usepackage{lmodern}
\usepackage{iftex}
\ifPDFTeX
  \usepackage[T1]{fontenc}
  \usepackage[utf8]{inputenc}
  \usepackage{textcomp} % provide euro and other symbols
\else % if luatex or xetex
  \usepackage{unicode-math}
  \defaultfontfeatures{Scale=MatchLowercase}
  \defaultfontfeatures[\rmfamily]{Ligatures=TeX,Scale=1}
\fi
% Use upquote if available, for straight quotes in verbatim environments
\IfFileExists{upquote.sty}{\usepackage{upquote}}{}
\IfFileExists{microtype.sty}{% use microtype if available
  \usepackage[]{microtype}
  \UseMicrotypeSet[protrusion]{basicmath} % disable protrusion for tt fonts
}{}
\makeatletter
\@ifundefined{KOMAClassName}{% if non-KOMA class
  \IfFileExists{parskip.sty}{%
    \usepackage{parskip}
  }{% else
    \setlength{\parindent}{0pt}
    \setlength{\parskip}{6pt plus 2pt minus 1pt}}
}{% if KOMA class
  \KOMAoptions{parskip=half}}
\makeatother
\usepackage{xcolor}
\IfFileExists{xurl.sty}{\usepackage{xurl}}{} % add URL line breaks if available
\IfFileExists{bookmark.sty}{\usepackage{bookmark}}{\usepackage{hyperref}}
\hypersetup{
  pdftitle={Problem Set 7: Estimation of TEs Using Matching Methods},
  pdfauthor={Tessie Dong, Derek Li, Andi Liu},
  colorlinks=true,
  linkcolor={Maroon},
  filecolor={Maroon},
  citecolor={Blue},
  urlcolor={Blue},
  pdfcreator={LaTeX via pandoc}}
\urlstyle{same} % disable monospaced font for URLs
\usepackage[margin=1cm]{geometry}
\usepackage{color}
\usepackage{fancyvrb}
\newcommand{\VerbBar}{|}
\newcommand{\VERB}{\Verb[commandchars=\\\{\}]}
\DefineVerbatimEnvironment{Highlighting}{Verbatim}{commandchars=\\\{\}}
% Add ',fontsize=\small' for more characters per line
\usepackage{framed}
\definecolor{shadecolor}{RGB}{248,248,248}
\newenvironment{Shaded}{\begin{snugshade}}{\end{snugshade}}
\newcommand{\AlertTok}[1]{\textcolor[rgb]{0.94,0.16,0.16}{#1}}
\newcommand{\AnnotationTok}[1]{\textcolor[rgb]{0.56,0.35,0.01}{\textbf{\textit{#1}}}}
\newcommand{\AttributeTok}[1]{\textcolor[rgb]{0.77,0.63,0.00}{#1}}
\newcommand{\BaseNTok}[1]{\textcolor[rgb]{0.00,0.00,0.81}{#1}}
\newcommand{\BuiltInTok}[1]{#1}
\newcommand{\CharTok}[1]{\textcolor[rgb]{0.31,0.60,0.02}{#1}}
\newcommand{\CommentTok}[1]{\textcolor[rgb]{0.56,0.35,0.01}{\textit{#1}}}
\newcommand{\CommentVarTok}[1]{\textcolor[rgb]{0.56,0.35,0.01}{\textbf{\textit{#1}}}}
\newcommand{\ConstantTok}[1]{\textcolor[rgb]{0.00,0.00,0.00}{#1}}
\newcommand{\ControlFlowTok}[1]{\textcolor[rgb]{0.13,0.29,0.53}{\textbf{#1}}}
\newcommand{\DataTypeTok}[1]{\textcolor[rgb]{0.13,0.29,0.53}{#1}}
\newcommand{\DecValTok}[1]{\textcolor[rgb]{0.00,0.00,0.81}{#1}}
\newcommand{\DocumentationTok}[1]{\textcolor[rgb]{0.56,0.35,0.01}{\textbf{\textit{#1}}}}
\newcommand{\ErrorTok}[1]{\textcolor[rgb]{0.64,0.00,0.00}{\textbf{#1}}}
\newcommand{\ExtensionTok}[1]{#1}
\newcommand{\FloatTok}[1]{\textcolor[rgb]{0.00,0.00,0.81}{#1}}
\newcommand{\FunctionTok}[1]{\textcolor[rgb]{0.00,0.00,0.00}{#1}}
\newcommand{\ImportTok}[1]{#1}
\newcommand{\InformationTok}[1]{\textcolor[rgb]{0.56,0.35,0.01}{\textbf{\textit{#1}}}}
\newcommand{\KeywordTok}[1]{\textcolor[rgb]{0.13,0.29,0.53}{\textbf{#1}}}
\newcommand{\NormalTok}[1]{#1}
\newcommand{\OperatorTok}[1]{\textcolor[rgb]{0.81,0.36,0.00}{\textbf{#1}}}
\newcommand{\OtherTok}[1]{\textcolor[rgb]{0.56,0.35,0.01}{#1}}
\newcommand{\PreprocessorTok}[1]{\textcolor[rgb]{0.56,0.35,0.01}{\textit{#1}}}
\newcommand{\RegionMarkerTok}[1]{#1}
\newcommand{\SpecialCharTok}[1]{\textcolor[rgb]{0.00,0.00,0.00}{#1}}
\newcommand{\SpecialStringTok}[1]{\textcolor[rgb]{0.31,0.60,0.02}{#1}}
\newcommand{\StringTok}[1]{\textcolor[rgb]{0.31,0.60,0.02}{#1}}
\newcommand{\VariableTok}[1]{\textcolor[rgb]{0.00,0.00,0.00}{#1}}
\newcommand{\VerbatimStringTok}[1]{\textcolor[rgb]{0.31,0.60,0.02}{#1}}
\newcommand{\WarningTok}[1]{\textcolor[rgb]{0.56,0.35,0.01}{\textbf{\textit{#1}}}}
\usepackage{graphicx}
\makeatletter
\def\maxwidth{\ifdim\Gin@nat@width>\linewidth\linewidth\else\Gin@nat@width\fi}
\def\maxheight{\ifdim\Gin@nat@height>\textheight\textheight\else\Gin@nat@height\fi}
\makeatother
% Scale images if necessary, so that they will not overflow the page
% margins by default, and it is still possible to overwrite the defaults
% using explicit options in \includegraphics[width, height, ...]{}
\setkeys{Gin}{width=\maxwidth,height=\maxheight,keepaspectratio}
% Set default figure placement to htbp
\makeatletter
\def\fps@figure{htbp}
\makeatother
\setlength{\emergencystretch}{3em} % prevent overfull lines
\providecommand{\tightlist}{%
  \setlength{\itemsep}{0pt}\setlength{\parskip}{0pt}}
\setcounter{secnumdepth}{-\maxdimen} % remove section numbering
\usepackage{amsmath}
\usepackage{optidef}
\usepackage{accents}
\usepackage{caption}
\ifLuaTeX
  \usepackage{selnolig}  % disable illegal ligatures
\fi

\begin{document}
\maketitle

\hypertarget{part-1-before-and-after-and-difference-in-difference-estimators-50-points}{%
\subsection{Part 1: Before and After and Difference in Difference
Estimators (50
Points)}\label{part-1-before-and-after-and-difference-in-difference-estimators-50-points}}

\noindent \textcolor{Maroon}{\textbf{Background}. In Pset 4 you estimated the average treatment effect of the offer of training for the NSW treated units in 1978 using the  Treated Control Comparison (TCC) estimator, the Regression Adjusted TCC estimator, and the Double Machine Learning (DML) estimator. The latter two estimators attempted to control for observable confounders, namely, observed pre-treatment differences between the NSW treated sample and the PSID-1 sample. Here we focus on methods that control for unobservable confounders, namely, the Before After Comparison Estimator and the Difference in Differences Estimators.}\textbackslash{}

\noindent \textcolor{Maroon}{\textbf{Objective}: You use the \texttt{nswpsid.csv} dataset to estimate the treatment effect (TE) of the offer of training via \textcolor{ForestGreen}{regression-based approaches} associated with the following specifications of the outcome equation:
\begin{eqnarray}
re78_{i} &=&\alpha +\rho D_{i}+u_{i}\text{, }i=1,...,2675\text{,}
\label{TCcomp} \\
earns_{i,t} &=& \alpha + \rho D78_{t}+u_{i}, \forall i \text{ with } D_i=1, t \in \{75,78\}  \label{BAfter} \\
earns_{i,78}-earns_{i,75} &=& \alpha + \rho D_i + u_{i}, i=1,\ldots,2675, t \in \{75,78\}  \label{FD} \\
earns_{i,t} &=& \alpha + \rho D_{i,t}+ \mu_i + \delta_t + u_{i,t}, i=1,\ldots,2675, t \in \{75,78\}   \label{TWFE} \\
earns_{i,t} &=& \alpha + \delta {D78}_{t} + \gamma D_i + \rho {D78}_t \times D_i +  u_{i,t}, i=1,\ldots,2675, t \in \{75,78\}  \label{DinD} 
\end{eqnarray}
\noindent Subscript $i$ denotes an individual. With reference to the original data (i.e., the data in wide format): 1) $re78_{i}$ represents the data field \texttt{re78}; 2) $D_{i}$ represents the data field \texttt{treat}. With reference to the long format data: 1) $earns_{i,t}$ captures the original content of the fields \texttt{re78} and \texttt{re75}, i.e., earnings of individual $i$ in year $t$; 2) ${D78}_{t}$ represent an indicator variable that equals 1 in the post-treatment year 1978 and zero in the pre-treatment year 1975. Table \ref{tab:reg-specs}'s column [1] references the regression specification. Column [2] gives the name of the approach. Column [3] indicates the regression coefficient of interest. You filled the first row's columns [4] and [5] in Pset4. Here you complete the rest of the table.}

\definecolor{maroon(html/css)}{rgb}{0.5, 0.0, 0.0}{}
\colorlet{lightmaroon}{Maroon!10}
\begin{table}[ht!]
\centering
\colorbox{lightmaroon}{
{\color{Maroon}
\begin{tabular}{ccccc}
\hline
\textbf{Reference} & \multicolumn{1}{c}{\textbf{Name of the }} & \textbf{Parameter} & \multicolumn{1}{c}{\textbf{Estimate}} & \textbf{SE} \\
\textbf{Model} & \multicolumn{1}{c}{\textbf{Estimation Approach}}            & \textbf{of Interest} &  &  \\ \hline
[1] & [2] & [3] & [4] & [5]  \\ \hline
expression (\ref{TCcomp})                 & Treatment-Control Comparison (TCC)                    & $\rho$             & -\$15.204.8  &  \$655.67  \\
expression (\ref{BAfter})                 & Before After Comparison (BA) & $\rho$             &  &  \\
expression (\ref{FD})                 & Difference-in-Differences via First Difference (FD) & $\rho$             &  &  \\
expression (\ref{TWFE})                 & Difference-in-Differences via Least Square Dummy Variable (LSDV) & $\rho$             &  &  \\
expression (\ref{TWFE})                 & Difference-in-Differences via Two-way Fixed Effects (TWFE) & $\rho$             &  &  \\
expression (\ref{DinD})                 & Difference-in-Differences (DD) & $\rho$             &  &  \\
\hline
\end{tabular}}}
\caption{\textcolor{Maroon}{Treatment Effect Estimates Based on Five Regression-Based Approaches Applied to Observational Data.}}
\label{tab:reg-specs}
\end{table}

\newpage

\begin{enumerate}
\def\labelenumi{\arabic{enumi}.}
\tightlist
\item
  (12 p) Use the specification in expression (2) to obtain the
  \textcolor{ForestGreen}{Before-After (BA) Estimator} of the treatment
  effect of the offer of training. Specifically:
\end{enumerate}

\begin{enumerate}
\def\labelenumi{\alph{enumi}.}
\tightlist
\item
  (4 p) Reshape the dataframe from \textcolor{ForestGreen}{wide format}
  to \textcolor{ForestGreen}{long format} so that for each sample
  individual there are 2 rows, one row for 1975 (pre-treatment) and one
  for 1978 (post-treatment). Verify that the reshaped dataframe has
  \(5,350\) rows.
  \textcolor{gray}{\textbf{Programming Guidance:} The original dataframe is in wide format because each row pertains to one individual, and columns \texttt{re75} and \texttt{re78} store the individual's earnings in 1975 and 1978. The long (i.e., panel) format has 2 rows for each individual (one for 1975, one for 1978), and 1 column storing the earnings for the corresponding year. We named the earnings column \texttt{earns}. Also, we added two columns: a) \texttt{dyear2} for $D78_{t}$, i.e., takes value 1 when the year is 1978 and 0 otherwise; b) \texttt{tdyear2} is the product of \texttt{dyear2} and \texttt{treat} and corresponds to $D78_t \times D_i$. To reshape a dataframe to long, use \href{https://tidyr.tidyverse.org/reference/gather.html}{\texttt{tidyr::gather( )}}. Find a worked out example \href{https://uc-r.github.io/tidyr}{here}. Or use \texttt{data.table::melt()}.}\label{item:BAfter-reshape}
\end{enumerate}

\begin{Shaded}
\begin{Highlighting}[]
\NormalTok{load\_data }\OtherTok{\textless{}{-}} \ControlFlowTok{function}\NormalTok{(filename)\{}
\NormalTok{  dt }\OtherTok{\textless{}{-}}\NormalTok{ data.table}\SpecialCharTok{::}\FunctionTok{as.data.table}\NormalTok{(}
\NormalTok{    readr}\SpecialCharTok{::}\FunctionTok{read\_csv}\NormalTok{(filename,}
                     \AttributeTok{col\_names =} \ConstantTok{TRUE}\NormalTok{,}
                     \AttributeTok{col\_types =}\NormalTok{ readr}\SpecialCharTok{::}\FunctionTok{cols}\NormalTok{(}
                       \AttributeTok{treat =}\NormalTok{ readr}\SpecialCharTok{::}\FunctionTok{col\_integer}\NormalTok{(),}
                       \AttributeTok{age =}\NormalTok{ readr}\SpecialCharTok{::}\FunctionTok{col\_integer}\NormalTok{(),}
                       \AttributeTok{edu =}\NormalTok{ readr}\SpecialCharTok{::}\FunctionTok{col\_integer}\NormalTok{(),}
                       \AttributeTok{black =}\NormalTok{ readr}\SpecialCharTok{::}\FunctionTok{col\_integer}\NormalTok{(),}
                       \AttributeTok{hisp =}\NormalTok{ readr}\SpecialCharTok{::}\FunctionTok{col\_integer}\NormalTok{(),}
                       \AttributeTok{married =}\NormalTok{ readr}\SpecialCharTok{::}\FunctionTok{col\_integer}\NormalTok{(),}
                       \AttributeTok{re74 =}\NormalTok{ readr}\SpecialCharTok{::}\FunctionTok{col\_double}\NormalTok{(),}
                       \AttributeTok{re75 =}\NormalTok{ readr}\SpecialCharTok{::}\FunctionTok{col\_double}\NormalTok{(),}
                       \AttributeTok{re78 =}\NormalTok{ readr}\SpecialCharTok{::}\FunctionTok{col\_double}\NormalTok{(),}
                       \AttributeTok{u74 =}\NormalTok{ readr}\SpecialCharTok{::}\FunctionTok{col\_integer}\NormalTok{(),}
                       \AttributeTok{u75 =}\NormalTok{ readr}\SpecialCharTok{::}\FunctionTok{col\_integer}\NormalTok{(),}
                       \AttributeTok{nodegree =}\NormalTok{ readr}\SpecialCharTok{::}\FunctionTok{col\_integer}\NormalTok{()}
\NormalTok{                     ))}
\NormalTok{  )}
  \FunctionTok{return}\NormalTok{(dt)}
\NormalTok{\}}
\end{Highlighting}
\end{Shaded}

\begin{Shaded}
\begin{Highlighting}[]
\NormalTok{filename\_psid }\OtherTok{=} \StringTok{"starter{-}files/nswpsid.csv"}

\NormalTok{dt\_psid }\OtherTok{\textless{}{-}} \FunctionTok{load\_data}\NormalTok{(}\AttributeTok{filename =}\NormalTok{ filename\_psid)}
\end{Highlighting}
\end{Shaded}

\begin{Shaded}
\begin{Highlighting}[]
\NormalTok{long\_psid }\OtherTok{\textless{}{-}}\NormalTok{ dt\_psid }\SpecialCharTok{\%\textgreater{}\%} \FunctionTok{gather}\NormalTok{(dyear2, earns, re75}\SpecialCharTok{:}\NormalTok{re78) }
\NormalTok{long\_psid}\SpecialCharTok{$}\NormalTok{dyear2 }\OtherTok{\textless{}{-}} \FunctionTok{ifelse}\NormalTok{(long\_psid}\SpecialCharTok{$}\NormalTok{dyear2 }\SpecialCharTok{==} \StringTok{\textquotesingle{}re75\textquotesingle{}}\NormalTok{, }\DecValTok{0}\NormalTok{, }\DecValTok{1}\NormalTok{)}
\NormalTok{long\_psid}\SpecialCharTok{$}\NormalTok{tdyear2 }\OtherTok{\textless{}{-}}\NormalTok{ long\_psid}\SpecialCharTok{$}\NormalTok{dyear2 }\SpecialCharTok{*}\NormalTok{ long\_psid}\SpecialCharTok{$}\NormalTok{earns}
\end{Highlighting}
\end{Shaded}

\begin{Shaded}
\begin{Highlighting}[]
\FunctionTok{nrow}\NormalTok{(long\_psid)}
\end{Highlighting}
\end{Shaded}

\begin{verbatim}
## [1] 5350
\end{verbatim}

We confirm that the reshaped dataframe has 5350 rows.

\begin{enumerate}
\def\labelenumi{\alph{enumi}.}
\setcounter{enumi}{1}
\tightlist
\item
  (1 p) Take the reshaped dataframe created in
  \textbf{\ref{item:BAfter-reshape}}; keep only the rows pertaining to
  the treated individuals, and check that the resulting dataframe has
  \(370\) rows because
  \(370=185 \text{ treated individuals } \times 2 \text{ years for each individual}\).\label{item:BAfter-filter}
\end{enumerate}

\begin{Shaded}
\begin{Highlighting}[]
\NormalTok{treatment }\OtherTok{\textless{}{-}}\NormalTok{ long\_psid }\SpecialCharTok{\%\textgreater{}\%} \FunctionTok{filter}\NormalTok{(treat }\SpecialCharTok{==} \DecValTok{1}\NormalTok{)}
\FunctionTok{nrow}\NormalTok{(treatment)}
\end{Highlighting}
\end{Shaded}

\begin{verbatim}
## [1] 370
\end{verbatim}

We confirm that after filtering on treatment, the resulting dataframe
has 370 rows.

\begin{enumerate}
\def\labelenumi{\alph{enumi}.}
\setcounter{enumi}{2}
\tightlist
\item
  (2 p) Use the dataframe created in \textbf{\ref{item:BAfter-filter}}
  and \texttt{stats::lm( )} to estimate \(\rho\) in the specification of
  expression (\ref{BAfter}).\label{item:BAfter-rho}
\end{enumerate}

\begin{Shaded}
\begin{Highlighting}[]
\NormalTok{pscore\_formula }\OtherTok{=} \FunctionTok{as.formula}\NormalTok{(earns }\SpecialCharTok{\textasciitilde{}}\NormalTok{ dyear2)}
\NormalTok{ols }\OtherTok{\textless{}{-}} \FunctionTok{lm}\NormalTok{(pscore\_formula, }\AttributeTok{data =}\NormalTok{ treatment)}
\end{Highlighting}
\end{Shaded}

\begin{table}[!htbp] \centering 
  \caption{Linear Regression Implementation of Specification 2} 
  \label{item:lms2} 
\begin{tabular}{@{\extracolsep{5pt}}lc} 
\\[-1.8ex]\hline 
\hline \\[-1.8ex] 
 & \multicolumn{1}{c}{\textit{Dependent variable:}} \\ 
\cline{2-2} 
\\[-1.8ex] & earns \\ 
\hline \\[-1.8ex] 
 dyear2 & 4,817.090$^{***}$ \\ 
  & (624.974) \\ 
  & \\ 
 Constant & 1,532.056$^{***}$ \\ 
  & (441.923) \\ 
  & \\ 
\hline \\[-1.8ex] 
Observations & 370 \\ 
R$^{2}$ & 0.139 \\ 
Adjusted R$^{2}$ & 0.137 \\ 
Residual Std. Error & 6,010.808 (df = 368) \\ 
F Statistic & 59.408$^{***}$ (df = 1; 368) \\ 
\hline 
\hline \\[-1.8ex] 
\textit{Note:}  & \multicolumn{1}{r}{$^{*}$p$<$0.1; $^{**}$p$<$0.05; $^{***}$p$<$0.01} \\ 
\end{tabular} 
\end{table}

We estimate \(\rho\) to be 4,817.090 as witnessed in Table
\ref{item:lms2}.

\newpage

\begin{enumerate}
\def\labelenumi{\alph{enumi}.}
\setcounter{enumi}{3}
\tightlist
\item
  (1 p) Verify that \(\hat{\rho}\) obtained in
  \textbf{\ref{item:BAfter-rho}} is numerically identical to
  \(\left( \overline{earn}_{78}^{1}-\overline{earns}_{75}^{1}\right)\),
  that is, it equals the difference between average post-treatment and
  average pre-treatment earnings using only the 185 individuals in the
  treatment sample.\label{item:BAfter-diff}
\end{enumerate}

\begin{Shaded}
\begin{Highlighting}[]
\NormalTok{pretreat75 }\OtherTok{\textless{}{-}}\NormalTok{ treatment }\SpecialCharTok{\%\textgreater{}\%} \FunctionTok{filter}\NormalTok{(dyear2 }\SpecialCharTok{==} \DecValTok{0}\NormalTok{)}
\NormalTok{posttreat78 }\OtherTok{\textless{}{-}}\NormalTok{ treatment }\SpecialCharTok{\%\textgreater{}\%} \FunctionTok{filter}\NormalTok{(dyear2 }\SpecialCharTok{==} \DecValTok{1}\NormalTok{)}

\FunctionTok{print}\NormalTok{(}\FunctionTok{mean}\NormalTok{(posttreat78}\SpecialCharTok{$}\NormalTok{earns) }\SpecialCharTok{{-}} \FunctionTok{mean}\NormalTok{(pretreat75}\SpecialCharTok{$}\NormalTok{earns))}
\end{Highlighting}
\end{Shaded}

\begin{verbatim}
## [1] 4817.09
\end{verbatim}

We verify that \(\hat{\rho}\) is numerically identical to
\(\left( \overline{earn}_{78}^{1}-\overline{earns}_{75}^{1}\right)\).

\begin{enumerate}
\def\labelenumi{\alph{enumi}.}
\setcounter{enumi}{4}
\tightlist
\item
  (4 p) Explain why the BA approach may or may not identify the average
  effect for the treated in the post-treatment year, i.e.,
  \({ATT}_{78}\).
  \textcolor{gray}{\textbf{Hint}: Use the result in \textbf{\ref{item:BAfter-diff}} to think about what this approach uses to identify the mean of potential outcomes w/ and w/out treatment and for different (sub)population. Feel free to utilize what we did in class.}
\end{enumerate}

Whilst we have the realized outcome of post-period earnings on the
treated units allowing us to find \(\overline{earn}_{78}^{1}\) as the
sample analogue for \(\mathbb{E}[y_{1i78} \mid E_i = 1]\) (where \(E_i\)
is \(D_{i75} = 0, D_{i78} = 1\)), we are unable tounable to learn the
potential outcomes of post-period earnings if those same units had not
been treated (thus unable to find a sample analogue for
\(\mathbb{E}[y_{0i78} \mid E_i = 1]\)).

In order to produce an estimate for \({ATT}_{78}\) we choose to assume
that the counterfactual for treated units are the same as earnings in
the pre-treatment period, so that
\(\mathbb{E}[y_{0i78} \mid E_i = 1] = \mathbb{E}[y_{0i75} \mid E_i = 1]\).
This allows us to take \(\overline{earns}_{75}^{1}\) as the sample
analogue.

However, the BA approach might not identity the average effect for the
treated in the post-treatment year \(ATT_{78}\) because there could be
unaccounted biases for the computed \(ATT_{78}\) if
\(\mathbb{E}[y_{0i78} \mid E_i = 1] \neq \mathbb{E}[y_{0i75} \mid E_i = 1]\).
In which case, \(ATT_{78}\) would be a biased estimator.

\end{document}
