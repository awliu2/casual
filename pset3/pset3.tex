% Options for packages loaded elsewhere
\PassOptionsToPackage{unicode}{hyperref}
\PassOptionsToPackage{hyphens}{url}
\PassOptionsToPackage{dvipsnames,svgnames,x11names}{xcolor}
%
\documentclass[
]{article}
\usepackage{amsmath,amssymb}
\usepackage{iftex}
\ifPDFTeX
  \usepackage[T1]{fontenc}
  \usepackage[utf8]{inputenc}
  \usepackage{textcomp} % provide euro and other symbols
\else % if luatex or xetex
  \usepackage{unicode-math} % this also loads fontspec
  \defaultfontfeatures{Scale=MatchLowercase}
  \defaultfontfeatures[\rmfamily]{Ligatures=TeX,Scale=1}
\fi
\usepackage{lmodern}
\ifPDFTeX\else
  % xetex/luatex font selection
\fi
% Use upquote if available, for straight quotes in verbatim environments
\IfFileExists{upquote.sty}{\usepackage{upquote}}{}
\IfFileExists{microtype.sty}{% use microtype if available
  \usepackage[]{microtype}
  \UseMicrotypeSet[protrusion]{basicmath} % disable protrusion for tt fonts
}{}
\makeatletter
\@ifundefined{KOMAClassName}{% if non-KOMA class
  \IfFileExists{parskip.sty}{%
    \usepackage{parskip}
  }{% else
    \setlength{\parindent}{0pt}
    \setlength{\parskip}{6pt plus 2pt minus 1pt}}
}{% if KOMA class
  \KOMAoptions{parskip=half}}
\makeatother
\usepackage{xcolor}
\usepackage[margin=1in]{geometry}
\usepackage{color}
\usepackage{fancyvrb}
\newcommand{\VerbBar}{|}
\newcommand{\VERB}{\Verb[commandchars=\\\{\}]}
\DefineVerbatimEnvironment{Highlighting}{Verbatim}{commandchars=\\\{\}}
% Add ',fontsize=\small' for more characters per line
\usepackage{framed}
\definecolor{shadecolor}{RGB}{248,248,248}
\newenvironment{Shaded}{\begin{snugshade}}{\end{snugshade}}
\newcommand{\AlertTok}[1]{\textcolor[rgb]{0.94,0.16,0.16}{#1}}
\newcommand{\AnnotationTok}[1]{\textcolor[rgb]{0.56,0.35,0.01}{\textbf{\textit{#1}}}}
\newcommand{\AttributeTok}[1]{\textcolor[rgb]{0.13,0.29,0.53}{#1}}
\newcommand{\BaseNTok}[1]{\textcolor[rgb]{0.00,0.00,0.81}{#1}}
\newcommand{\BuiltInTok}[1]{#1}
\newcommand{\CharTok}[1]{\textcolor[rgb]{0.31,0.60,0.02}{#1}}
\newcommand{\CommentTok}[1]{\textcolor[rgb]{0.56,0.35,0.01}{\textit{#1}}}
\newcommand{\CommentVarTok}[1]{\textcolor[rgb]{0.56,0.35,0.01}{\textbf{\textit{#1}}}}
\newcommand{\ConstantTok}[1]{\textcolor[rgb]{0.56,0.35,0.01}{#1}}
\newcommand{\ControlFlowTok}[1]{\textcolor[rgb]{0.13,0.29,0.53}{\textbf{#1}}}
\newcommand{\DataTypeTok}[1]{\textcolor[rgb]{0.13,0.29,0.53}{#1}}
\newcommand{\DecValTok}[1]{\textcolor[rgb]{0.00,0.00,0.81}{#1}}
\newcommand{\DocumentationTok}[1]{\textcolor[rgb]{0.56,0.35,0.01}{\textbf{\textit{#1}}}}
\newcommand{\ErrorTok}[1]{\textcolor[rgb]{0.64,0.00,0.00}{\textbf{#1}}}
\newcommand{\ExtensionTok}[1]{#1}
\newcommand{\FloatTok}[1]{\textcolor[rgb]{0.00,0.00,0.81}{#1}}
\newcommand{\FunctionTok}[1]{\textcolor[rgb]{0.13,0.29,0.53}{\textbf{#1}}}
\newcommand{\ImportTok}[1]{#1}
\newcommand{\InformationTok}[1]{\textcolor[rgb]{0.56,0.35,0.01}{\textbf{\textit{#1}}}}
\newcommand{\KeywordTok}[1]{\textcolor[rgb]{0.13,0.29,0.53}{\textbf{#1}}}
\newcommand{\NormalTok}[1]{#1}
\newcommand{\OperatorTok}[1]{\textcolor[rgb]{0.81,0.36,0.00}{\textbf{#1}}}
\newcommand{\OtherTok}[1]{\textcolor[rgb]{0.56,0.35,0.01}{#1}}
\newcommand{\PreprocessorTok}[1]{\textcolor[rgb]{0.56,0.35,0.01}{\textit{#1}}}
\newcommand{\RegionMarkerTok}[1]{#1}
\newcommand{\SpecialCharTok}[1]{\textcolor[rgb]{0.81,0.36,0.00}{\textbf{#1}}}
\newcommand{\SpecialStringTok}[1]{\textcolor[rgb]{0.31,0.60,0.02}{#1}}
\newcommand{\StringTok}[1]{\textcolor[rgb]{0.31,0.60,0.02}{#1}}
\newcommand{\VariableTok}[1]{\textcolor[rgb]{0.00,0.00,0.00}{#1}}
\newcommand{\VerbatimStringTok}[1]{\textcolor[rgb]{0.31,0.60,0.02}{#1}}
\newcommand{\WarningTok}[1]{\textcolor[rgb]{0.56,0.35,0.01}{\textbf{\textit{#1}}}}
\usepackage{graphicx}
\makeatletter
\def\maxwidth{\ifdim\Gin@nat@width>\linewidth\linewidth\else\Gin@nat@width\fi}
\def\maxheight{\ifdim\Gin@nat@height>\textheight\textheight\else\Gin@nat@height\fi}
\makeatother
% Scale images if necessary, so that they will not overflow the page
% margins by default, and it is still possible to overwrite the defaults
% using explicit options in \includegraphics[width, height, ...]{}
\setkeys{Gin}{width=\maxwidth,height=\maxheight,keepaspectratio}
% Set default figure placement to htbp
\makeatletter
\def\fps@figure{htbp}
\makeatother
\setlength{\emergencystretch}{3em} % prevent overfull lines
\providecommand{\tightlist}{%
  \setlength{\itemsep}{0pt}\setlength{\parskip}{0pt}}
\setcounter{secnumdepth}{-\maxdimen} % remove section numbering
\usepackage{amsmath}
\ifLuaTeX
  \usepackage{selnolig}  % disable illegal ligatures
\fi
\IfFileExists{bookmark.sty}{\usepackage{bookmark}}{\usepackage{hyperref}}
\IfFileExists{xurl.sty}{\usepackage{xurl}}{} % add URL line breaks if available
\urlstyle{same}
\hypersetup{
  pdftitle={Problem Set 3},
  pdfauthor={Tessie Dong, Derek Li, Andi Liu},
  colorlinks=true,
  linkcolor={Maroon},
  filecolor={Maroon},
  citecolor={Blue},
  urlcolor={Blue},
  pdfcreator={LaTeX via pandoc}}

\title{Problem Set 3}
\author{Tessie Dong, Derek Li, Andi Liu}
\date{Due Jan 18th, 2024}

\begin{document}
\maketitle

\textbf{Part 1: Describe the Data} \newline Combine the NSW data in
\texttt{nswre74\_control.csv} and \texttt{nswre74\_treated.csv} and
complete Table \ref{tab:Tab_NSW_1}. Note that variables \texttt{1-10}
are \(\color{ForestGreen}{predetermined}\), i.e., capture
characteristics determined at or before treatment assignment; some of
these variables are background characteristics (e.g., \texttt{edu}),
others capture a subject's pre-RCT labor market experience (e.g.,
\texttt{u75}). \texttt{re78} is the observed outcome variable.
\texttt{treat} is the indicator of treatment status.

\begin{Shaded}
\begin{Highlighting}[]
\CommentTok{\# load and combine the data sets into one dataframe}
\NormalTok{df1 }\OtherTok{\textless{}{-}}\NormalTok{ utils}\SpecialCharTok{::}\FunctionTok{read.csv}\NormalTok{(}\AttributeTok{file =} \StringTok{"nswre74\_control.csv"}\NormalTok{)}
\NormalTok{df2 }\OtherTok{\textless{}{-}}\NormalTok{ utils}\SpecialCharTok{::}\FunctionTok{read.csv}\NormalTok{(}\AttributeTok{file =} \StringTok{"nswre74\_treated.csv"}\NormalTok{)}
\NormalTok{df }\OtherTok{\textless{}{-}} \FunctionTok{rbind}\NormalTok{(df1, df2)}
\end{Highlighting}
\end{Shaded}

\begin{Shaded}
\begin{Highlighting}[]
\CommentTok{\# count units in each sample}
\NormalTok{dplyr}\SpecialCharTok{::}\FunctionTok{tally}\NormalTok{(dplyr}\SpecialCharTok{::}\FunctionTok{group\_by}\NormalTok{(df, treat))}
\end{Highlighting}
\end{Shaded}

\begin{verbatim}
## # A tibble: 2 x 2
##   treat     n
##   <int> <int>
## 1     0   260
## 2     1   185
\end{verbatim}

\begin{Shaded}
\begin{Highlighting}[]
\CommentTok{\# generate mean summary statistics for each variable and treatment}
\NormalTok{dplyr}\SpecialCharTok{::}\FunctionTok{summarise\_all}\NormalTok{((dplyr}\SpecialCharTok{::}\FunctionTok{group\_by}\NormalTok{(df, treat)), }\FunctionTok{list}\NormalTok{(mean))}
\end{Highlighting}
\end{Shaded}

\begin{verbatim}
## # A tibble: 2 x 12
##   treat   age   edu black   hisp married nodegree  re74  re75  re78   u74   u75
##   <int> <dbl> <dbl> <dbl>  <dbl>   <dbl>    <dbl> <dbl> <dbl> <dbl> <dbl> <dbl>
## 1     0  25.1  10.1 0.827 0.108    0.154    0.835 2107. 1267. 4555. 0.75  0.685
## 2     1  25.8  10.3 0.843 0.0595   0.189    0.708 2096. 1532. 6349. 0.708 0.6
\end{verbatim}

A completed version of Table 1 is provided on the following page.
\newpage

\begin{table}[!ht!]
\centering
\begin{tabular}{ccccc}
\hline
\textbf{Variable} & \textbf{Variable}    & \textbf{Variable}    & \multicolumn{2}{c}{\textbf{Sample Average}}    \\ \cline{4-5} 
\textbf{Counter} & \textbf{Name}       & \textbf{Definition}                        & \textbf{Treated} & \textbf{Control} \\ \hline
1 & \texttt{age}      & Age in years                      & 25.8        & 25.1       \\
2 & \texttt{edu}      & Education in years                & 10.3        & 10.1       \\
3 & \texttt{nodegree} & 1 if education $<12$              & 0.708       & 0.835      \\
4 & \texttt{black}    & 1 if Black                        & 0.843       & 0.827      \\
5 & \texttt{hisp}     & 1 if Hispanic                     & 0.0595      & 0.108      \\
6 & \texttt{married}  & 1 if married                      & 0.189       & 0.154      \\
7 & \texttt{u74}      & 1 if unemployed in '74            & 0.708       & 0.75       \\
8 & \texttt{u75}      & 1 if unemployed in '75            & 0.6         & 0.685      \\
9 & \texttt{re74}     & Real earnings in '74 (in '82 \$)  & 2096        & 2107       \\
10 & \texttt{re75}    & Real earnings in '75 (in '82 \$)  & 1532        & 1267       \\
\hline 
11 & \texttt{re78}    & Real earnings in '78 (in '82 \$)  & 6349        & 4555       \\
12 & \texttt{treat}   & 1 if received offer of training   & 1           & 0          \\
\hline
Sample Size                                             &&& 185         & 260              \\ \hline
\end{tabular}
\caption{Descriptive statistics for the NSW data by group.}
\label{tab:Tab_NSW_1}
\end{table}

\textbf{Part 2: Test Balance}

\begin{enumerate}
\def\labelenumi{\arabic{enumi}.}
\tightlist
\item
  Test \textcolor{ForestGreen}{balance} for each of the 10 OPVs in Table
  (\ref{tab:Tab_NSW_1}), i.e., test that each variable's mean is the
  same in the control and treated groups. Do so by running 10 simple
  linear regressions (SLR) specifications. Use a 5\% significance level
  and look at the relevant 10 t-tests, comment on your findings.
  \textcolor{gray}{\textbf{Hint}: Each OPV is the dependent variable in its regression equation. All 10 SLR models have the same covariates.}
\end{enumerate}

\begin{Shaded}
\begin{Highlighting}[]
\NormalTok{ols\_p\_values }\OtherTok{\textless{}{-}} \FunctionTok{list}\NormalTok{()}
\NormalTok{ols\_coefficients }\OtherTok{\textless{}{-}} \FunctionTok{list}\NormalTok{()}
\NormalTok{ols\_se }\OtherTok{\textless{}{-}} \FunctionTok{list}\NormalTok{()}
\NormalTok{vars }\OtherTok{\textless{}{-}} \FunctionTok{names}\NormalTok{(df)[}\DecValTok{2}\SpecialCharTok{:}\DecValTok{12}\NormalTok{]}
\ControlFlowTok{for}\NormalTok{ (var }\ControlFlowTok{in}\NormalTok{ vars) \{}
\NormalTok{    formula }\OtherTok{\textless{}{-}}\NormalTok{ stats}\SpecialCharTok{::}\FunctionTok{formula}\NormalTok{(}\FunctionTok{paste}\NormalTok{(var, }\StringTok{"\textasciitilde{}treat"}\NormalTok{))}
\NormalTok{    lm\_model }\OtherTok{\textless{}{-}} \FunctionTok{lm}\NormalTok{(}\AttributeTok{formula =}\NormalTok{ formula, }\AttributeTok{data =}\NormalTok{ df)}
\NormalTok{    ols\_p\_values[[var]] }\OtherTok{\textless{}{-}} \FunctionTok{summary}\NormalTok{(lm\_model)}\SpecialCharTok{$}\NormalTok{coefficients[}\DecValTok{2}\NormalTok{, }\DecValTok{4}\NormalTok{]}
\NormalTok{    ols\_coefficients[[var]] }\OtherTok{\textless{}{-}} \FunctionTok{summary}\NormalTok{(lm\_model)}\SpecialCharTok{$}\NormalTok{coefficients[}\DecValTok{2}\NormalTok{, }\DecValTok{1}\NormalTok{]}
\NormalTok{    ols\_se[[var]] }\OtherTok{\textless{}{-}} \FunctionTok{summary}\NormalTok{(lm\_model)}\SpecialCharTok{$}\NormalTok{coefficients[}\DecValTok{2}\NormalTok{, }\DecValTok{2}\NormalTok{]}
\NormalTok{\}}
\CommentTok{\# print(ols\_p\_values)}
\CommentTok{\# print(ols\_coefficients)}
\CommentTok{\# print(ols\_se)}
\end{Highlighting}
\end{Shaded}

\begin{table}
\centering
\begin{tabular}{ccccc}
\hline
& \textbf{Variable} & \textbf{Coefficient} & \textbf{se} & \textbf{p-value} \\ \hline
1 & \texttt{age} & 0.7623701 & 0.6827511 & 0.264764 \\
2 & \texttt{edu} & 0.2574844 & 0.1721353 & 0.135411 \\
3 & \texttt{nodegree} & -0.1265073 & 0.0393452 & 0.001398 \\
4 & \texttt{black} & 0.01632017 & 0.03588617 & 0.649493 \\
5 & \texttt{hisp} & -0.04823285& 0.0271632 & 0.076474 \\
6 & \texttt{married} & 0.03534304 & 0.03604844 & 0.327408 \\
7 & \texttt{u74} & -0.04189189& 0.0426221 & 0.326209 \\
8 & \texttt{u75} & -0.08461538& 0.04582176 & 0.065469 \\
9 & \texttt{re74} & -11.45296 & 516.478 & 0.982318 \\
10 & \texttt{re75} & 265.1463 & 303.1555 & 0.382254 \\ \hline
\end{tabular}
\caption{Coefficients, standard errors, and p-values for the t-tests of balance for each OPV.}
\end{table}

Table 2 shows the p-values for the t-tests of balance for each OPV.
\newline  We can see that for most of our variables, the p-value is
greater than 0.05, so we fail to reject the null hypothesis that the
means are the same in the control and treated groups. However, for
\texttt{nodegree} we reject the null hypothesis that the means are the
same in the control and treated groups, and we cannot claim that these
groups are balanced for this variable.

\newpage

\begin{enumerate}
\def\labelenumi{\arabic{enumi}.}
\setcounter{enumi}{1}
\tightlist
\item
  Testing balance as done in Q1 suffers from the so called
  \href{https://en.wikipedia.org/wiki/Multiple_comparisons_problem}{\textcolor{ForestGreen}{``multiple comparisons''}}
  or \textcolor{ForestGreen}{``multiple testing'' problem} which occurs
  when one considers a set of statistical inferences simultaneously. The
  \href{https://explainxkcd.com/wiki/index.php/882:_Significant}{problem}
  emerges because as more variables are compared, it becomes more likely
  that the treatment and control groups appear to differ on at least one
  attribute \textit{by random chance alone}. To deal with this problem
  we use an estimation methodology called
  \textcolor{ForestGreen}{SUR estimation} and then test just one
  hypothesis, the \textit{joint} hypothesis that all OPVs are balanced,
  i.e., their means are the same in the two groups. SUR stands for
  \textcolor{ForestGreen}{seemingly unrelated regression} and it is a
  special case of
  \textcolor{ForestGreen}{feasible Generalized Least Squares (GLS) estimation}.
  Instead of estimating the coefficients \textit{equation-by-equation}
  by OLS (and done in Q1 you combine the 10 equations in a system of
  equations and estimate the coefficients present in all the equations
  jointly, accounting for the fact that the unobservables may be
  correlated across equations within an individual (we continue to
  assume that they are uncorrelated across units). After estimation, you
  use standard testing procedures to test the \textit{joint} hypothesis
  that the slope coefficients in all the equations of the SUR system are
  zero. This joint test is a test of covariate balance that does not
  suffer from the ``multiple testing'\,' problem.
\end{enumerate}

\begin{enumerate}
\def\labelenumi{\alph{enumi}.}
\tightlist
\item
  Estimate the SUR system. Are the estimated coefficients and their SEs
  different from those obtained in Q1? Comment.
  \textcolor{gray}{\textbf{Hint}: In 2 situations there is no efficiency payoff to GLS versus OLS: 1) the unobservables are uncorrelated across equations within an individual; and 2) the equations have identical covariates.}
\end{enumerate}

\begin{Shaded}
\begin{Highlighting}[]
\CommentTok{\# create a list of formulas for each equation}
\NormalTok{formulas }\OtherTok{\textless{}{-}} \FunctionTok{list}\NormalTok{()}
\NormalTok{vars }\OtherTok{\textless{}{-}} \FunctionTok{names}\NormalTok{(df)[}\DecValTok{2}\SpecialCharTok{:}\DecValTok{12}\NormalTok{]}
\ControlFlowTok{for}\NormalTok{ (var }\ControlFlowTok{in}\NormalTok{ vars) \{}
\NormalTok{  formulas[[var]] }\OtherTok{\textless{}{-}} \FunctionTok{formula}\NormalTok{(}\FunctionTok{paste}\NormalTok{(var, }\StringTok{"\textasciitilde{}treat"}\NormalTok{))}
\NormalTok{\}}
\end{Highlighting}
\end{Shaded}

\begin{Shaded}
\begin{Highlighting}[]
\CommentTok{\# estimate the SUR model}
\NormalTok{sur\_fit }\OtherTok{\textless{}{-}}\NormalTok{ systemfit}\SpecialCharTok{::}\FunctionTok{systemfit}\NormalTok{(}\AttributeTok{formula =}\NormalTok{ formulas, }\AttributeTok{data =}\NormalTok{ df, }\AttributeTok{method =} \StringTok{"SUR"}\NormalTok{)}
\end{Highlighting}
\end{Shaded}

\begin{Shaded}
\begin{Highlighting}[]
\CommentTok{\# print the coefficients and SE of the SUR model}
\NormalTok{i }\OtherTok{\textless{}{-}} \DecValTok{1}
\ControlFlowTok{while}\NormalTok{ (i }\SpecialCharTok{\textless{}=} \DecValTok{11}\NormalTok{)\{}
  \FunctionTok{print}\NormalTok{(}\FunctionTok{sprintf}\NormalTok{(}\StringTok{"\%s: \%f, \%f"}\NormalTok{,}
\NormalTok{                vars[i],}
                \FunctionTok{summary}\NormalTok{(sur\_fit)}\SpecialCharTok{$}\NormalTok{coefficients[i }\SpecialCharTok{*} \DecValTok{2}\NormalTok{, }\DecValTok{1}\NormalTok{],}
                \FunctionTok{summary}\NormalTok{(sur\_fit)}\SpecialCharTok{$}\NormalTok{coefficients[i }\SpecialCharTok{*} \DecValTok{2}\NormalTok{, }\DecValTok{2}\NormalTok{]))}
\NormalTok{  i }\OtherTok{\textless{}{-}}\NormalTok{ (i }\SpecialCharTok{+} \DecValTok{1}\NormalTok{)}
\NormalTok{\}}
\end{Highlighting}
\end{Shaded}

\begin{verbatim}
## [1] "age: 0.762370, 0.682751"
## [1] "edu: 0.257484, 0.172135"
## [1] "black: 0.016320, 0.035886"
## [1] "hisp: -0.048233, 0.027163"
## [1] "married: 0.035343, 0.036048"
## [1] "nodegree: -0.126507, 0.039345"
## [1] "re74: -11.452958, 516.477971"
## [1] "re75: 265.146299, 303.155497"
## [1] "re78: 1794.342382, 632.853392"
## [1] "u74: -0.041892, 0.042622"
## [1] "u75: -0.084615, 0.045822"
\end{verbatim}

\begin{Shaded}
\begin{Highlighting}[]
\CommentTok{\# print the coefficients and SE of the OLS model}
\ControlFlowTok{for}\NormalTok{ (var }\ControlFlowTok{in}\NormalTok{ vars)}
\NormalTok{\{}
  \FunctionTok{print}\NormalTok{(}\FunctionTok{sprintf}\NormalTok{(}\StringTok{"\%s: \%f, \%f "}\NormalTok{, var, ols\_coefficients[var], ols\_se[var]))}
\NormalTok{\}}
\end{Highlighting}
\end{Shaded}

\begin{verbatim}
## [1] "age: 0.762370, 0.682751 "
## [1] "edu: 0.257484, 0.172135 "
## [1] "black: 0.016320, 0.035886 "
## [1] "hisp: -0.048233, 0.027163 "
## [1] "married: 0.035343, 0.036048 "
## [1] "nodegree: -0.126507, 0.039345 "
## [1] "re74: -11.452958, 516.477971 "
## [1] "re75: 265.146299, 303.155497 "
## [1] "re78: 1794.342382, 632.853392 "
## [1] "u74: -0.041892, 0.042622 "
## [1] "u75: -0.084615, 0.045822 "
\end{verbatim}

The estimated coefficients and SEs are the same across OLS and GLS
estimations for all the OPVs. This result may be due to the fact that
the equations have the same covariate, i.e.~\texttt{treat}.

\begin{enumerate}
\def\labelenumi{\alph{enumi}.}
\setcounter{enumi}{1}
\tightlist
\item
  Test \textit{joint} balance by testing the
  \textcolor{ForestGreen}{joint hypothesis} that the coefficients of
  \texttt{treat} are zero in all the equations of the system. Spell out
  null and alternative hypotheses, comment on findings, and verify the
  test's value and p-value ``manually.'\,'
  \textcolor{gray}{\textbf{Hint}: Use the \href{https://en.wikipedia.org/wiki/Likelihood-ratio_test}{\textcolor{ForestGreen}{Likelihood Ratio (LR) test}} where the unrestricted model is the system of equations with \texttt{treat} as covariate, and the restricted model is the system with no regression covariates, i.e., with only the constant term. To verify the value of the test use its algebraic expression. Recall that the test has a $\chi^{2}_{df}$ distribution with $df$ = number of restrictions.}
\end{enumerate}

\begin{Shaded}
\begin{Highlighting}[]
\CommentTok{\# create a list of formulas null\_system with only the constant}
\NormalTok{null\_formulas }\OtherTok{\textless{}{-}} \FunctionTok{list}\NormalTok{()}
\ControlFlowTok{for}\NormalTok{ (var }\ControlFlowTok{in}\NormalTok{ vars) \{}
\NormalTok{  null\_formulas[[var]] }\OtherTok{\textless{}{-}} \FunctionTok{formula}\NormalTok{(}\FunctionTok{paste}\NormalTok{(var, }\StringTok{"\textasciitilde{}1"}\NormalTok{))}
\NormalTok{\}}

\CommentTok{\# pass the null\_system to systemfit}
\NormalTok{null\_fit }\OtherTok{\textless{}{-}}\NormalTok{ systemfit}\SpecialCharTok{::}\FunctionTok{systemfit}\NormalTok{(}\AttributeTok{formula =}\NormalTok{ null\_formulas, }\AttributeTok{data =}\NormalTok{ df, }\AttributeTok{method =} \StringTok{"SUR"}\NormalTok{)}

\CommentTok{\# calculate the LR test statistic}
\NormalTok{lrtest\_obj }\OtherTok{\textless{}{-}}\NormalTok{ lmtest}\SpecialCharTok{::}\FunctionTok{lrtest}\NormalTok{(null\_fit, sur\_fit)}
\NormalTok{lr\_statistic }\OtherTok{\textless{}{-}}\NormalTok{ lrtest\_obj}\SpecialCharTok{$}\NormalTok{Chisq[}\DecValTok{2}\NormalTok{]}
\NormalTok{lr\_df }\OtherTok{\textless{}{-}}\NormalTok{ lrtest\_obj}\SpecialCharTok{$}\NormalTok{Df[}\DecValTok{2}\NormalTok{]}

\NormalTok{p\_value }\OtherTok{\textless{}{-}}\NormalTok{ stats}\SpecialCharTok{::}\FunctionTok{pchisq}\NormalTok{(lr\_statistic, }\AttributeTok{df =}\NormalTok{ lr\_df, }\AttributeTok{lower.tail =} \ConstantTok{FALSE}\NormalTok{)}
\FunctionTok{print}\NormalTok{(}\FunctionTok{sprintf}\NormalTok{(}\StringTok{"LR test statistic: \%f, p{-}value: \%f"}\NormalTok{, lr\_statistic, p\_value))}
\end{Highlighting}
\end{Shaded}

\begin{verbatim}
## [1] "LR test statistic: 27.021898, p-value: 0.004560"
\end{verbatim}

The null hypothesis is that the coefficients of \texttt{treat} are zero
in all the equations of the system. The alternative hypothesis is that
there is at least one equation with a nonzero coefficient, implying that
the control and treatment froups have different means for an OPV.

The p-value of this test is significant (p-value \textless{} 0.05). This
result shows that we can reject null hypothesis that the coefficients of
\texttt{treat} are zero in all the equations of the system, implying
that the assignment of treatment did not balance all the subjects'
characteristics.

\begin{enumerate}
\def\labelenumi{\arabic{enumi}.}
\setcounter{enumi}{2}
\item
  Test that the OPVs do not predict treatment assignment. Spell out null
  and alternative hypotheses, comment on findings, and verify the test's
  value and p-value ``manually.'\,' Why would scientists carry out this
  test?
  \textcolor{gray}{\textbf{Hint:} Use a MLRM and the \textcolor{ForestGreen}{F-test for the overall significance of the regression}. \textbf{Programming guidance}: Use \texttt{summary()} after estimation, e.g., \texttt{summary(lm\_fit)\$fstatistic} returns the test's value and degrees of freedom. Use \href{https://stat.ethz.ch/R-manual/R-devel/library/stats/html/Fdist.html}{\texttt{stats::pf()}} to verify the test's p-value.}

  We want to test that, given the covariates, the treatment assignment
  is random. We can do this by testing that the coefficients of the
  covariates are zero in the regression of treatment assignment on the
  covariates. \newline

  Given \((Y, X^{OPV})\) where \(Y\) represents the treatment
  assignment, and \(X\) is a vector of our OPV covariates, we can test
  the following model: \(Y = \beta'X^{OPV} + \epsilon\) with these
  hypotheses: \newline  \[ 
   \begin{cases}
   H_0: \beta = 0 \\
   H_1: \beta \neq 0 
   \end{cases}
   \] In which \(\beta\) is a vector of coefficients for the OPV
  covariates. Thus, we want \(\beta = 0\), i.e the OPV covariates all
  have no correlation to the treatment \(Y\).
\end{enumerate}

\begin{Shaded}
\begin{Highlighting}[]
\CommentTok{\# create a list of formulas for each equation}

\NormalTok{lm\_fit }\OtherTok{\textless{}{-}} \FunctionTok{lm}\NormalTok{(}\AttributeTok{formula =} \FunctionTok{formula}\NormalTok{(}\FunctionTok{paste}\NormalTok{(}\StringTok{"treat"}\NormalTok{, }\StringTok{"\textasciitilde{}"}\NormalTok{, }\FunctionTok{paste}\NormalTok{(vars, }\AttributeTok{collapse =} \StringTok{"+"}\NormalTok{))), }\AttributeTok{data =}\NormalTok{ df)}
\CommentTok{\# summary(lm\_fit)}
\CommentTok{\# summary(lm\_fit)$fstatistic}
\end{Highlighting}
\end{Shaded}


\end{document}
