\noindent \textcolor{Maroon}{\textbf{Background: Common Support and Trimming}. Let $\widehat{p}_i$ denote the estimate of the pscore for unit $i$. Let $\left(\widehat{p}^{C,l},\widehat{p}^{C,u}\right)$ denote the smallest (``lower'') and largest (``upper'') $\widehat{p}_{i}$ among control units and $\left(\widehat{p}^{T,l},\widehat{p}^{T,u}\right) $ the smallest and largest $\widehat{p}_{i}$ among treated units. The range $\left[\max \left\{ \widehat{p}^{C,l},\widehat{p}^{T,l}\right\},\min \left\{ \widehat{p}^{C,u},\widehat{p}^{T,u}\right\} \right] $ is called the \textcolor{ForestGreen}{common support}. To understand this terminology observe that $\left(\widehat{p}^{C,l},\widehat{p}^{C,u}\right)$ are the extremes of the support of the empirical distribution of the pscore among the controls and $\left(\widehat{p}^{T,l},\widehat{p}^{T,u}\right) $ are the extremes of the support of the empirical distribution of the pscore among the treated; the common support is the intersection of the two empirical supports. By definition, a treated unit whose pscore is outside of the common support has no exact match among the control units and viceversa. \textcolor{ForestGreen}{Trimming} is a procedure whereby we drop sample units. There are various ways of implementing trimming for the purpose of pscore matching. A common approach is to drop all units (control and treated alike) whose $\widehat{p}_{i}$ falls outside of the common support.}
